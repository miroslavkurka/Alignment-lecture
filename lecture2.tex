\documentclass{beamer}
\usepackage{graphicx}
\usepackage{listings}
\usepackage{xcolor}
\usepackage{amsmath}
\usepackage{caption}
\captionsetup[figure]{font=tiny}

\title{Central Dogma of Molecular Biology}
\author{M. Kurka} 
\institute{github.com/miroslavkurka/Alignment-lecture}
\usetheme{Frankfurt}
\usecolortheme{spruce}
\setbeamertemplate{itemize/enumerate body begin}{\small}

\begin{document}


\maketitle

\section{Overview}
\begin{frame}
    \frametitle{Overview}
    \begin{itemize}
        \item DNA
        \item Protein
        \item Sequence
        \item The Dogma
        \item Conclusion
    \end{itemize}
\end{frame}
\section{DNA}
\begin{frame}

\frametitle{What is DNA?}
\begin{exampleblock}{DNA}
    DNA is long linear polymer that carry information in a form that can be passed from one generation to the next. DNA consist of a large number of linked nucleotides, each
composed of a sugar, a phosphate, and a base. Sugars linked by phosphates form a common backbone that plays a structural role, whereas the sequence of bases along a nucleic acid strand carries genetic information.
\end{exampleblock}
\begin{exampleblock}{}
    nucleotides $\approx$ bytes 
\end{exampleblock}
\end{frame}

\section{Protein}
\begin{frame}
\frametitle{What is a Protein?}
\begin{exampleblock}{Protein}
    Proteins are linear polymers built of monomer units called amino acids. The sequence of linked amino acids is called the primary structure. Proteins spontaneously fold up into three-dimensional structures that are determined by the sequence of amino acids in the protein polymer. Three-dimensional structure formed by hydrogen bonds between amino acids near one another is called secondary structure, whereas tertiary structure is formed by long-range interactions between amino acids. Protein function depends directly on this three- dimensional structure 
\end{exampleblock}
\begin{exampleblock}{}
protein $\approx$ function
    
\end{exampleblock}
\end{frame}
\section{Sequence}
\begin{frame}[fragile]
\frametitle{What are Sequences?}

\begin{exampleblock}{DNA sequence}
$$X  = x_1 x_2...x_n, \; x_i \in \{ C,A,G,T \} $$
\end{exampleblock}

\end{frame}

\section{Dogma}
\begin{frame}[fragile]
\frametitle{Central Dogma}
\begin{exampleblock}{}
DNA makes RNA, and RNA makes protein
\end{exampleblock}

\begin{figure}
\centering
\includegraphics[scale=0.4]{stryer.png}
\caption[scale=0.4]{Page from Berg JM, Tymoczko JL, Stryer L. Biochemistry, describing the dogma.}

\end{figure}
\end{frame}


\begin{frame}[fragile]
\frametitle{The Actual Dogma}
\begin{exampleblock}{}
    {\large ``The central dogma of molecular biology deals with the detailed
    residue-by-residue transfer of sequential information. It states
    that such information can not be transferred from protein to either
    protein or nucleic acid. ''}
    \vskip5mm
    \hspace*\fill{\small-- Francis Crick, Central Dogma of Molecular Biology}\footnotemark[1]\
\end{exampleblock}
\footnotetext[1]{CRICK, F. Central Dogma of Molecular Biology. Nature 227, 561–563 (1970).}
\begin{itemize}
    \item Postulated in 1958 paper
    \item Restated by Crick in 1970 as a response to Temin's paper\footnote[2]{TEMIN, H. Central Dogma Reversed. Nature 226, 1198–1199 (1970)}
    \item The common misconception comes from Watson's 1965 text
\end{itemize}
\end{frame}
\begin{frame}[fragile]
\begin{figure}
\centering
\includegraphics[scale=0.6]{1958.png}
\caption{Archived copy of the 1958 paper from Cold Spring Harbor}
\end{figure}
\end{frame}
\begin{frame}
\frametitle{The Actual Dogma}
First divided into 3 categories 
\begin{itemize}
    \item Empirically adequate: DNA $\rightarrow$ DNA, DNA $\rightarrow$ RNA, RNA $\rightarrow$ Protein, RNA $\rightarrow$ RNA
    \item Presumed (due to RNA viruses): RNA $\rightarrow$ DNA, DNA $\rightarrow$ Protein
    \item No evidence nor theoretical requirement: Protein $\rightarrow$ Protein, Protein $\rightarrow$ RNA, Protein $\rightarrow$ DNA 
\end{itemize}
\begin{figure}
\includegraphics[scale=0.4]{1970.png}
\end{figure}
\end{frame}
\begin{frame}[fragile]
\frametitle{Three Categories Now\footnote{1970}}
We have three categories defined as followed
\begin{itemize}
    \item General Transfer 
    \item Special Transfer
    \item Unknown Transfer
\end{itemize}
\begin{figure}
\includegraphics[scale=0.7]{1.png}
\includegraphics[scale=0.7]{2.png}
\includegraphics[scale=0.7]{3.png}
\caption{General, Special and Unkown Transfer respectively}
    
\end{figure}

\end{frame}
\begin{frame}
    \begin{exampleblock}{}
        So far, however, there is no evidence for the
        first two of these except in a cell infected with an RNA
        virus. In such a cells the central dogma demands that at
        least one of the first two special transfers should occur this statement, incidentally, shows the power of the
        central dogma in making theoretical predictions.
    \end{exampleblock}
\end{frame}

\end{document}